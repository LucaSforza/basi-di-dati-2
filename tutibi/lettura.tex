\documentclass{article}

\begin{document}
    TuTubi deve consentire la registrazione degli utenti, di cui interessa il nome e la data
    di iscrizione. Gli utenti registrati possono pubblicare video, visualizzare quelli disponibili,
    oltre che esprimere su di essi valutazioni e commenti testuali.
    \section{Specifica classe Video}
    In particolare, di un video pubblicato da un utente interessa conoscere il titolo, la
    durata e la descrizione (oltre che il nome del file dove questo viene memorizzato da
    parte del sistema).
    Inoltre, di ogni video interessa conoscere la categoria (unica) a
    cui appartiene, oltre che un insieme di parole chiave dette tag (almeno una) che ne
    descrivano il contenuto in modo più strutturato. Delle categorie e dei tag interessa il
    nome.
    \section{Risposta ai video in modo consistente}
    TuTubi deve anche permettere ad un utente di pubblicare un nuovo video segnalando
    che si tratta di una risposta ad un video già esistente (utile per temi come la politica dove
    vi possono essere posizioni diverse che stimolano un dibattito). TuTubi deve mantenere
    tale informazione in modo consistente, garantendo che nessun utente possa pubblicare
    un video in risposta ad un video pubblicato da sé stesso.
    \section{Cronologia e visualizzazioni}
    TuTubi deve poi fornire un servizio di cronologia, ricordando la sequenza di tutti i
    video visionati da ogni singolo utente, con relativa data e ora. Inoltre, per ogni video
    interessa conoscere il numero complessivo di volte che è stato visionato.
    \section{valutazioni}
    Gli utenti di TuTubi devono inoltre avere la facoltà di esprimere una valutazione
    (un valore da 0 \emph{pessimo} a 5 \emph{ottimo}) per ogni video che visionano. Tali valutazioni
    serviranno poi al sistema per promuovere i video più belli (cfr. seguito) e scovare i più
    brutti. Tuttavia, per evitare degenerazioni nel meccanismo delle votazioni, l’utente che
    ha pubblicato un video non può votarlo, mentre ogni altro utente può votarlo al più una
    volta, indipendentemente dal numero di volte che l’ha visionato. In ogni caso però deve
    essere impossibile per un utente votare un video che non ha mai visionato.
    \section{Commenti}
    Inoltre, per favorire uno spirito di comunità tra gli utenti del servizio, si prevede che
    questi possano esprimere commenti testuali ai video che visionano (dei quali interessa
    anche data e ora). A differenza delle valutazioni, un utente può esprimere più commenti
    per lo stesso video. Anche qui, non deve essere permesso ad un utente di scrivere
    commenti per video che non ha mai visionato.
    \section{playlist}
    Ogni utente può creare delle playlist personali, ovvero delle collezioni ordinate di
    video che gradisce vedere, oppure vuole condividere con altri utenti. Le playlist infatti
    (di cui interessa il nome e la data di creazione) possono essere pubbliche o private: solo
    le playlist pubbliche possono essere visualizzate dagli altri utenti. A tal fine, il sistema
    deve permettere ad ogni utente di ottenere le playlist pubbliche di un altro utente a sua
    scelta.
    \section{Use Case Utente}
    TuTubi deve permettere ad un utente di iscriversi, pubblicare nuovi video, creare e
    modificare le sue playlist, ed esprimere valutazioni e commenti sui video che visiona.
    Inoltre, TuTubi deve consentire la ricerca di video: in particolare, data una categoria,
    un insieme di tag, ed un intero v tra 0 e 5, si vogliono restituire tutti i video disponibili
    di quella categoria che posseggono almeno uno tra i tag indicati, e che abbiano una
    valutazione media di almeno v (se un video non ha ancora alcuna valutazione, deve
    essere restituito comunque). TuTubi deve poi permettere di cercare, data una categoria,
    i video di quella categoria che hanno il numero maggiore di video in risposta, al fine di
    isolare le discussioni più animate tra gli utenti.
    \section{Use Case Redazione}
    La redazione di TuTubi ha infine la facoltà di censurare dei video, ad esempio perché
    di contenuto coperto da copyright, osceno, ecc. Un video censurato non può essere né
    visionato, né votato, né commentato, né aggiunto ad alcuna playlist, né restituito come
    risultato di una ricerca. Un video, una volta censurato, non può tornare più visibile. Il
    motivo di una censura deve essere mantenuto nel sistema per usi interni.
\end{document}